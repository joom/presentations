\documentclass{beamer}
\usepackage{listings, bussproofs}

\usetheme{Warsaw}
\usecolortheme{wolverine}
\beamertemplatenavigationsymbolsempty

\title{Intro to Interactive Theorem Proving}
\author{Joomy Korkut}
\institute{Wesleyan University}
\date{October 5th, 2017}

\begin{document}

\begin{frame}
\titlepage
\end{frame}

\begin{frame}{Why?}
\begin{itemize}
\item<1-> Proofs are too long to write formally!
\item<2-> Proofs are too long to check carefully!
\item<3-> Seeing what we proved so far helps during the proof process.
\end{itemize}
\end{frame}

\begin{frame}{Examples}
\begin{itemize}
  \item<1-> Four color theorem (required checking 1,936 cases)
  \\ Appel \& Haken 1976 with computer assistance
  \\ Werner \& Gonthier 2015 with Coq
  \item<2-> Kepler conjecture (hexagonal close packing of spheres)
  \\ Hales 1998 with computer assistance (C++ etc.)
  \\ Hales 2014 with HOL Light and Isabelle (Flyspeck)
\end{itemize}
\end{frame}

\begin{frame}{How?}
\begin{itemize}
\item<1-> Curry-Howard isomorphism
\item<2-> Calculus of inductive constructions
\end{itemize}
\end{frame}

\begin{frame}{Curry-Howard Isomorphism}
  for simply typed $\lambda$-calculus and intuitionistic prop logic,
  \begin{itemize}
    \item Types are propositions!
    \pause
    \item Terms are proofs!
  \end{itemize}

  \pause
  \begin{columns}
    \begin{column}{0.5\textwidth}
      \begin{prooftree}
        \AxiomC{$\Gamma, A \vdash B$}
        \UnaryInfC{$\Gamma \vdash A \to B$}
      \end{prooftree}
    \end{column}
    \begin{column}{0.5\textwidth}  %%<--- here
      \begin{prooftree}
        \AxiomC{$\Gamma, x : A \vdash t : B$}
        \UnaryInfC{$\Gamma \vdash \lambda x.t : A \to B$}
      \end{prooftree}
    \end{column}
  \end{columns}
\end{frame}

\begin{frame}{Calculus of Inductive Constructions}
  \begin{itemize}
    \item Extension of Curry-Howard to an higher-order type theory based
    \item Allows $\forall$ and $\exists$ in your types
  \end{itemize}
  \pause
  \begin{alertblock}{Note}
  CIC is constructive, but you can add
  the law of excluded middle as an axiom and prove classical theorems.
  \end{alertblock}
\end{frame}

\begin{frame}{Proof Assistants}
  \begin{itemize}
    \item Isabelle (1986) (tactics!)
    \item Coq (1989) (tactics!)
    \item Agda (2007)
  \end{itemize}
\end{frame}
\end{document}
